\documentclass[a4paper,10pt]{article}

\usepackage[margin=3cm]{geometry}
\usepackage{cmap}
\usepackage[T2A]{fontenc}
\usepackage[utf8]{inputenc}
\usepackage[english, russian]{babel}
\usepackage{hyperref, array, xcolor, listings, amsmath, ragged2e, enumitem}
\usepackage{amsmath, amsfonts}
\usepackage{listings}

\title{Повторяем к коллоквиуму}
\date{}

\begin{document}
	\maketitle
	\tableofcontents
	\newpage
	\section{Билет 2}
	\begin{center} 
		\item \paragraph{Формулировка} 
	\end{center}
	$\mathbb{Q}$ - счётное, $\mathbb{R}$ - бесконечное несчётное \\
	\begin{center} 
		\item \paragraph{Идея доказательства} 
	\end{center}
	$\mathbb{Q}$ нумеруем "змейкой", для $\mathbb{R}$ рассматриваем промежуток $[0, 1)$, предполагаем, что он счётный, нумеруем бесконечные десятичные дроби, потом формируем новую бесконечную дробь, которая не равно ни одной из предыдущих
	\begin{center} 
		\item \paragraph{Дополнительные формулировки, теоремы} 
	\end{center}
	\textbf{Действительное число} - класс эквивалентности систем стягивающихся рациональных отрезков \\
	\textbf{Отношение неравенства между действительными числами} - $a>b := a + (-b) > 0$ \\
	\textbf{Свойство Архимеда} $(\forall a \in \mathbb{R}) (\exists n \in \mathbb{N}) \colon n > a$ или же $\forall a, b \in \mathbb{R}, a<b, \exists n \colon na>b$ \\
	\textbf{Плотность множества рациональных чисел во множестве действительных} $(\forall a \in \mathbb{R})(\forall b \in \mathbb{R}, a < b)(\exists c \in \mathbb{Q}) a < c < b$ \\
	\textbf{Операции с действительными числами} - определяем сложение и умножение через нового представителя (в виде системы бесконечных стягивающихся рациональных отрезков), а деление через умножение на обратное число \\
	\textbf{Представление действительных чисел бесконечными десятичными дробями} - если число не является ни целым, ни конечной дробью, то возьмём интервал $(a, a+1)$, куда попало число, потом будем последовательно делить интервал на 10 и рекурсивно искать цифры после запятой \\
	\section{Билет 3}
	\begin{center} 
		\item \paragraph{Формулировка} 
	\end{center}
		Любое ограниченное сверху (снизу) непустое множество $E \subset \mathbb{R}$ имеет точную верхнюю (нижнюю) грань
	\begin{center} 
		\item \paragraph{Идея доказательства} 
	\end{center}
		Выделим отдельно множество верхних граней и дополнение до множества действительных чисел. Из условия полноты найдется число между ними. Далее от противного доказываем, что c - верхняя грань и наименьшая верхняя грань \\
	\begin{center} 
		\item \paragraph{Дополнительные формулировки, теоремы} 
	\end{center}
	\textbf{Условие полноты действительных чисел} - идея доказательства: строим систему стягивающихся отрезков (по сути, множество десятичных дробей) в число $c$ (по лемме), потом проверяем условие от противного (найдем $ q\geq b$, но $q \le c$ \\
	\textbf{Лемма про стягивающиеся отрезки} - доказываем от противного, с эквивалентной системой отрезков. Потом приходим к противоречию при $p_{n_{0}}' - q_{n_{0}} > 0$ \\ 
	\textbf{Определение верхней (нижней) грани} \\
	\textbf{Определение точной грани} \\
	\textbf{Определение ограниченного множества} \\
	\section{Билет 4}
	\begin{center} 
		\item \paragraph{Формулировка} 
	\end{center}
	Сумма бесконечно малых последовательностей - бесконечно малая последовательность \\
	Произведение бесконечно малой последовательности на ограниченную - бесконечно малая последовательность \\
	\begin{center} 
		\item \paragraph{Идея доказательства} 
	\end{center}
	Расписываем по определению бесконечно малые последовательности, ограниченную последовательность, да и всё \\
	\begin{center} 
		\item \paragraph{Дополнительные формулировки, теоремы} 
	\end{center}
	\textbf{Бесконечно малая последовательность} \\
	\textbf{Бесконечно большая последовательность} \\
	\textbf{Обратная последовательность к бесконечно малой - бесконечно большая} - расписать по определению
	\section{Билет 5}
	\begin{center} 
		\item \paragraph{Формулировка} 
	\end{center}
	\begin{itemize}
		\item Ограниченность сходящейся последовательности
		\item Отделимость от нуля
		\item Переход к пределу в неравенстве
		\item О двух силовиках
	\end{itemize}
	\begin{center} 
		\item \paragraph{Идея доказательства} 
	\end{center}
	\begin{enumerate}
		\item Ограничиваем часть через предел, потом берём максимум/минимум из конечного числа элементов и ограничения из предела
		\item Берём за эпсилон $\frac{|l|}{2}$
		\item От противного, за эпсилон берём $\frac{l_{1}-l_{2}}{2}$
		\item Расписываем по определению
	\end{enumerate}
	\begin{center} 
		\item \paragraph{Дополнительные формулировки, теоремы} 
	\end{center}
	\textbf{Предел числовой последовательности} \\
	\textbf{Единственность предела} - доказываем от противного, берём эпсилон $\frac{l_{2} - l_{1}}{2}$ \\
	\section{Билет 6}
	\begin{center} 
		\item \paragraph{Формулировка} 
	\end{center}
	Неформально: пределы складываются/умножаются при сложении/умножении функций. Делятся, если $(\forall n \in \mathbb{N}) (y_{n} \neq 0)$ и предел не равен 0
	\begin{center} 
		\item \paragraph{Идея доказательства} 
	\end{center}
	\textbf{Сложение} - расписываем по определению, берём эпсилон $\frac{1}{2}$ \\
	\textbf{Умножение} - расписываем по определению, используем ограниченность одной из функций, подгоняем эпсилон \\
	\textbf{Деление} - расписываем по определению, используем отделимость от нуля, подгоняем эпсилон \\
	\section{Билет 7}
	\begin{center} 
		\item \paragraph{Формулировка} 
	\end{center}
	Теорема о пределе ограниченной монотонной последовательности (Вейерштрасса) \\
	Каждая неубывающая (невозрастающая) ограниченная сверху (снизу) последовательность имеет предел, причём её предел равен точной верхней (нижней) грани
	\begin{center} 
		\item \paragraph{Идея доказательства} 
	\end{center}
	Рассматриваем частный случай, расписываем по определению ограниченность, точную грань и монотонность
	\begin{center} 
		\item \paragraph{Дополнительные формулировки, теоремы} 
	\end{center}
	\textbf{Монотонность последовательности} \\
	\textbf{Любая монотоннаяя последовательность имеет предел в $\overline{\mathbb{R}}$ (расписать по определению)}
	\section{Билет 8}
	\begin{center} 
		\item \paragraph{Формулировка} 
	\end{center}
	Последовательность $x_{n} = (1+\frac{1}{n})^{n}$ сходится, её предел равен числу $e$
	\begin{center} 
		\item \paragraph{Идея доказательства} 
	\end{center}
	Рассматриваем последовательность со степенью $n+1$. Доказываем, что ограничена снизу и убывает, значит по Вейерштрасса имеет предел
	\begin{center} 
		\item \paragraph{Дополнительные формулировки, теоремы} 
	\end{center}
	\textbf{Неравенство Бернулли}
	\section{Билет 9}
	\begin{center} 
		\item \paragraph{Формулировка} 
	\end{center}
	Всякая последовательность вложенных отрезков имеет непустое пересечение
	\begin{center} 
		\item \paragraph{Идея доказательства} 
	\end{center}
	Вейерштрасс для левой и правой границы, потом переход к пределу в неравенстве
	\begin{center} 
		\item \paragraph{Дополнительные формулировки, теоремы} 
	\end{center}
	\textbf{Последовательность вложенных отрезков} \\
	\textbf{Поседовательность вложенных стягивающихся отрезков} \\
	\textbf{Последовательность вложенных стягивающихся отрезков стягивается к числу} \\
	\section{Билет 10}
	\begin{center} 
		\item \paragraph{Формулировка} 
	\end{center}
	Каждая ограниченная последовательность имеет конечный верхний и нижний предел \\
	\[ L = \varlimsup_{n \to \infty}x_{n} \] - верхний \\
	\[ l = \varliminf_{n \to \infty}x_{n} \] - нижний \\
	Справедливы следующие утверждения: \\
	1.1) $(\forall \epsilon > 0)(\exists N \in \mathbb{N})(\forall n > N) x_{n} < L + \epsilon \land (\forall \epsilon > 0)(\forall N \in \mathbb{N})(\exists n > N) x_{n} > L - \epsilon$ \\
	1.2) $(\forall \epsilon > 0)(\exists N \in \mathbb{N})(\forall n > N) x_{n} > l - \epsilon \land (\forall \epsilon > 0)(\forall N \in \mathbb{N})(\exists n > N) x_{n} < l + \epsilon$ \\
	2.1) $L = \lim_{n \to \infty} \sup{x_{n},x_{n+1},...}$\\
	2.2) $l = \lim_{n \to \infty} \inf{x_{n},x_{n+1},...}$
	\begin{center} 
		\item \paragraph{Идея доказательства} 
	\end{center}
	Докажем для L (для l аналогично)\\
	Рассмотрим вспомогательную последовательность супремумов $s_{n}$ \\
	Сначала докажем существование её предела по теореме Вейерштрасса \\
	Распишем по определению предел, и докажем 1.1.1 \\
	Возьмём произвольный член, представим его как верхнюю грань, распишем по определению точную верхнюю грань и докажем 1.1.2 \\
	Докажем, что L - частичный предел, для этого, используя 1.1.1 и 1.1.2 будем строить такую подпоследовательность, чтобы она стремилась к L (делим $\epsilon$ каждый раз) \\
	Докажем, что L - наибольший частичный предел, возьмём другой частичный предел для $x_{m_{i}}$, где $m_{i}$ - возрастающая последовательность, тогда напишем 1.1.1 для обеих последовательностей, потом предельный переход и готово
	\begin{center} 
		\item \paragraph{Дополнительные формулировки, теоремы} 
	\end{center}
	\textbf{Подпоследовательность} \\
	\textbf{Частичный предел}
	\section{Билет 11}
	\begin{center} 
		\item \paragraph{Формулировка} 
	\end{center}
	Из каждой ограниченной последовательности можно выделить сходящуюся подпоследовательность
	\begin{center} 
		\item \paragraph{Идея доказательства} 
	\end{center}
	Будем делить отрезок на два, а потом рекурсивно делить половину, в которой бесконечно много элементом, построим систему стягивающихся отрезков, который стягиваются в $c$. Построим подпоследовательность, выбирая элементы из этих половин.
	\begin{center} 
		\item \paragraph{Дополнительные формулировки, теоремы} 
	\end{center}
	\textbf{Подпоследовательность}
	\section{Билет 13}
	\begin{center} 
		\item \paragraph{Формулировка} 
	\end{center}
	Числовая последовательность сходится $\Leftrightarrow$ она фундаментальна
	\begin{center} 
		\item \paragraph{Идея доказательства} 
	\end{center}
	$\Rightarrow$ Расписываем предел по определению для $x_{n+p}$ и $x_{n}$ \\
	$\Leftarrow$ Доказываем, что последовательность ограниченная из определения фундаментальности с фиксированным $\epsilon$
	Выделяем по Больцано-Вейерштрасса сходящуюся подпоследовательность, потом расписываем по определению её предел и фундаментальность нашей последовательности, дальше неравенство треугольника
	\begin{center} 
		\item \paragraph{Дополнительные формулировки, теоремы} 
	\end{center}
	\textbf{Фундаментальная последовательность}
	\section{Билет 14}
	\begin{center} 
		\item \paragraph{Формулировка} 
	\end{center}
	Пусть функция $f$ определена в некоторой $\dot{U_{\delta_{0}}}(a), a \in \overline{\mathbb{R}} \cup {\infty}$, тогда
	\[ \lim_{x \to a} f(x) = l (l \in \overline{\mathbb{R}} \cup {\infty}) \], означает: определение по Коши, определение по Гейне. Определения эквивалентны
	\begin{center} 
		\item \paragraph{Идея доказательства} 
	\end{center}
	К $\Rightarrow$ Г \\
	Берём произвольную последовательность Гейне, пишем предел по определению, $x$ попал в $\dot{U_{\delta_{0}}}(a)$, из Коши берём ту $\delta$, которая нашлась для произвольного $\epsilon$ \\
	Г $\Rightarrow$ К \\
	Пишем отрицание Коши, фиксируем $\epsilon$, строим по нему последовательность Гейне, по Гейне пишем предел, но замечаем противоречие, т.к. каждое значение функции не попадает в $U_{\epsilon}$ (рассматриваем $a$ конечное и бесконечное)
	\begin{center} 
		\item \paragraph{Дополнительные формулировки, теоремы} 
	\end{center}
	\textbf{Проколотая окрестность} \\
	\textbf{Свойства пределов (как у последовательностей)} \\
	\textbf{Левосторонний, правосторонний предел} \\
	\textbf{Арифметические операции}
	\section{Билет 15}
	\begin{center} 
		\item \paragraph{Формулировка} 
	\end{center}
	Сходится $\Leftrightarrow$ фундаментальна
	\begin{center} 
		\item \paragraph{Идея доказательства} 
	\end{center}
	$\Rightarrow$ Расписываем по определению предел для двух точек из окрестности \\
	$\Leftarrow$ Пишем по определению предел последовательности Гейне, распишем фундаментальность через последовательность Гейне, по Коши получим предел. Докажем, что для любой последовательности Гейне тот же предел от противного (создав новую последовательность Гейне)
	\section{Билет 16}
	\begin{center} 
		\item \paragraph{Формулировка} 
	\end{center}
	2 случая монотонности, для каждого 2 односторонних предела
	\begin{center} 
		\item \paragraph{Идея доказательства} 
	\end{center}
	Докажем для частного случая - левосторонний предел для невозрастающей функции \\
	Рассматриваем случай, когда предел равен $-\infty$, расписываем это по определению, фиксируем $x$, расписываем левосторонний предел по Коши, подбираем $\delta$ такой, чтобы воспользоваться монотонностью \\
	Если предел больше $-\infty$, расписываем точную нижнюю грань через двойной неравенство, расписываем левосторонний предел по Коши, подбираем $\delta$ такой, чтобы воспользоваться монотонностью
	\begin{center} 
		\item \paragraph{Дополнительные формулировки, теоремы} 
	\end{center}
	\textbf{Существование одностороннего предела по Коши, Гейне} \\
	\textbf{Разные виды монотонности}
	\section{Билет 17}
	\begin{center} 
		\item \paragraph{Формулировка} 
	\end{center}
	Если \[ \lim_{x \to x_{0}(\pm 0)} f(x) = a \] и $g$ непрерывна в $a$, то \[ \lim_{x \to x_{0}(\pm 0)} g(f(x)) = g(a) \] 
	\begin{center} 
		\item \paragraph{Идея доказательства} 
	\end{center}
	Распишем по Гейне определение предела, расписываем по опреелению непрерывность $g$ в точке $a$
	\begin{center} 
		\item \paragraph{Дополнительные формулировки, теоремы} 
	\end{center}
	\textbf{Определение непрерывности} \\
	\textbf{Непрерывность слева} \\
	\textbf{Непрерывности справа} \\
	\textbf{Точка разрыва} \\
	\textbf{Точка разрыва первого рода} \\
	\textbf{Точка разрыва второго рода} \\
	\textbf{Точка устранимого разрыва} \\
	\textbf{Точка бесконечного разрыва} \\
	\textbf{Непрерывность по Гейне Коши} \\
	\section{Билет 18}
	\begin{center} 
		\item \paragraph{Формулировка} 
	\end{center}
	Если $f$ непрерывна на $[a, b]$, то $f$ ограничена на $[a, b]$
	\begin{center} 
		\item \paragraph{Идея доказательства} 
	\end{center}
	От противного, раписываем неограниченность, строим последовательность, исходя из ограниченности, выделяем сходящуюся подпоследовательность, противоречие по непрерывности
	\begin{center} 
		\item \paragraph{Дополнительные формулировки, теоремы} 
	\end{center}
	\textbf{Непрерывность на множестве}
	\section{Билет 19}
	\begin{center} 
		\item \paragraph{Формулировка} 
	\end{center}
	Если $f$ непрерывна на отрезке $[a, b]$, то $(\exists x',x'' \in [a,b]), f(x')=\sup_{x \in [a,b]}f(x'), f(x'')=\inf_{x \in [a,b]}f(x'')$
	\begin{center} 
		\item \paragraph{Идея доказательства} 
	\end{center}
	Докажем для $sup$, распишем по определению, строим последовательность, исходя из ограниченности, выделяем сходящуюся подпоследовательность, противоречие по непрерывности
	\section{Билет 20}
	\begin{center} 
		\item \paragraph{Формулировка} 
	\end{center}
	Теорема Больцано-Коши о промежуточных значениях непрерывной функции. Пусть $f$ непрерывна на $[a,b]$, $\forall c = f(x_{1}) < d = f(x_{2}), x_{1}, x_{2} \in [a,b]) \forall e \in (c,d) (\exists \gamma \in [a,b] f(\gamma) = e$
	\begin{center} 
		\item \paragraph{Идея доказательства} 
	\end{center}
	Р. частный случай $c < e = 0 < d$. Рассматриваем отрезок $[a_{1}, b_{1}], f(a_{1})*g(b_{1}) < 0$, делим пополам, берём половину с разными знаками, получаем систему стягивающихся отрезков, по принципу Кантора получим в пересечении какое-то число, которое равно пределу последовательностей правых и левых границ. Дальше по определению непрерывность и предельный переход
	\section{Билет 21}
	\begin{center} 
		\item \paragraph{Формулировка} 
	\end{center}
	Если f строго монотонна и непрерывна на промежутке $I = [a,b]$, то на промежутке $E = [f(a), f(b)]$ определена, строго монотонна в том же смысле, что и $f$ и непрерывна обратная функия $f^{-1}$
	\begin{center} 
		\item \paragraph{Идея доказательства} 
	\end{center}
	Доказываем, что существует обратная функция по 2 теореме Вейерштрасса, для доказательства строгой монотонности берём произвольные $y_{1}$ и                      $y_{2}$, например, и используем строгую монотонность $f$
	\begin{center} 
		\item \paragraph{Дополнительные формулировки, теоремы} 
	\end{center}
	\textbf{Обратная функция}
\end{document}
