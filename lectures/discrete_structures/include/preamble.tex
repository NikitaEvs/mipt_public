\usepackage[utf8]{inputenc}
\usepackage{cmap}
\usepackage[T2A]{fontenc}
\usepackage{textcomp}
\usepackage[russian]{babel}
\usepackage{url}
\usepackage{graphicx}
\usepackage{float}
\usepackage{booktabs}
\usepackage{enumitem}
\usepackage{soulutf8}

\usepackage{parskip}

\usepackage{emptypage}
\usepackage{subcaption}
\usepackage{multicol}
\usepackage{xcolor}

% Math stuff
\usepackage{amsmath, amsfonts, mathtools, amsthm, amssymb}
% Fancy script capitals
\usepackage{mathrsfs}
\usepackage{cancel}
% Bold math
\usepackage{bm}
% Some shortcuts
\newcommand\N{\ensuremath{\mathbb{N}}}
\newcommand\R{\ensuremath{\mathbb{R}}}
\newcommand\Z{\ensuremath{\mathbb{Z}}}
\renewcommand\O{\ensuremath{\emptyset}}
\newcommand\Q{\ensuremath{\mathbb{Q}}}
\renewcommand\C{\ensuremath{\mathbb{C}}}

%Make implies and impliedby shorter
\let\implies\Rightarrow
\let\impliedby\Leftarrow
\let\iff\Leftrightarrow
\let\epsilon\varepsilon

\usepackage{mdframed}
\mdfsetup{skipabove=1em,skipbelow=0em}

\newmdtheoremenv[nobreak=true]{theorem}{Теорема}[section]
\newmdtheoremenv[nobreak=true]{corollary}{Следствие}[theorem]
\newmdtheoremenv[nobreak=true]{lemma}[theorem]{Лемма}

\theoremstyle{remark}
\newtheorem*{remark}{Утверждение}

\theoremstyle{definition}
\newmdtheoremenv[nobreak=true]{definition}{Определение}[section]

\renewcommand\qedsymbol{$\blacksquare$}

\makeatletter
\def\thm@space@setup{%
  \thm@preskip=\parskip \thm@postskip=0pt
}

\usepackage{xifthen}


\def\@lecture{}%
\newcommand{\lecture}[2]{
    \ifthenelse{\isempty{#2}}{%
        \def\@lecture{Лекция #1}%
    }{%
        \def\@lecture{Лекция #1: #2}%
    }%
    \section{\@lecture}
}

\usepackage{geometry}  
\geometry{left=20mm,right=20mm,top=25mm,bottom=30mm} % задание полей текста
\setlength{\headheight}{18pt}
\usepackage{fancyhdr}
\pagestyle{fancy}
\renewcommand{\footrulewidth}{0.4pt}

% LE: left even
% RO: right odd
% CE, CO: center even, center odd

\fancyhead[RO,LE]{\hyperlink{table}{К содержанию}} % Right odd,  Left even
\fancyhead[RE,LO]{\@lecture}          % Right even, Left odd

\fancyfoot[RO,LE]{\thepage}  % Right odd,  Left even
\fancyfoot[RE,LO]{\CourseInfo}          % Right even, Left odd
\fancyfoot[C]{}

\makeatother

\usepackage{hyperref}
\usepackage{xcolor}
\hypersetup{			
	unicode=true,           
	pdfstartview=FitH,
	colorlinks=true,       	
	linkcolor=blue,         
	citecolor=green,        
	filecolor=magenta,      
	urlcolor=cyan,          
}

\usepackage{tcolorbox}

\usepackage{listings}
\lstset{ 
  backgroundcolor=\color{white},   % цвет фона
  basicstyle=\footnotesize,        % размер
  breakatwhitespace=false,         % проблемы с отступами
  breaklines=true,                 % автоматический переход на новую строку
  captionpos=b,                    % sets the caption-position to bottom
  commentstyle=\color{black},      % стиль комментариев
  escapeinside={\%*}{*)},          % для вставки латеха в код
  frame=single,	                   % рамочка вокруг
  keepspaces=true,                 % сохраняет пробелы в коде (чтобы был кодстайл)
  keywordstyle=\color{blue},       % стиль кода
  language=C++,                    % язык программирования!
  morekeywords={*,...},            % увеличение словаря
  numbers=left,                    % где будет нумерация строк (none, left, right)
  numbersep=8pt,                   % расстояние между нумерацией и кодом
  numberstyle=\color{gray},        % стиль нумерации
  rulecolor=\color{gray},          % цвет рамки
  showspaces=false,                % если тру, то заменяет пробелы на особые подчеркивания; конфликтует с 'showstringspaces'
  showstringspaces=false,          % особые подчеркивания вместо пробелов в строках
  showtabs=false,                  % подчеркивания вместо табов
  stepnumber=1,                    % как часто показывать нумерацию. Если 1, то нумеруется каждая строка
  stringstyle=\color{violet},      % цвет строк
  tabsize=2,	                       % дефолтный размер таба
  texcl=true,
  title=\lstname                   % вставляет название файла, если импорт
}

\makeatletter
\newcommand{\highlight}[1]{%
  \setbox\@tempboxa\hbox{#1}%
  \ifdim\wd\@tempboxa>\linewidth
    \noindent
    \colorbox{pink}{%
      \parbox{\dimexpr\linewidth-2\fboxsep}{#1}%
    }%
  \else
    \colorbox{pink}{#1}%
  \fi}%Highlighter.
\makeatother
