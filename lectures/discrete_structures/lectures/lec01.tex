\lecture{1}{Графы}

\begin{definition}
	\highlight{Граф} --- пара множеств $(V, E)$, где  $V$ --- множество вершин, $E$ --- множество ребер
\end{definition}

\begin{definition}
	\highlight{Ребро}  --- неупорядоченная пара двух различных вершин
\end{definition}

\begin{definition}
	\highlight{Мультиграф} --- возможны кратные ребра (такие, которые соединяют одинаковые вершины)
\end{definition}

\begin{definition}
	\highlight{Псевдограф} --- мультиграф с петлями
\end{definition}

\begin{definition}
	\highlight{Ориентированный граф} --- в паре ребёр важен порядок
\end{definition}

\begin{definition}
	\highlight{$U$ и  $V$ смежные}, если $\exists $ ребро, которое их соединяет 
\end{definition}

\begin{definition}
	\highlight{Вершина инцидента ребру}, т.е. является концом ребра
\end{definition}

\begin{remark}
	Существуют матрицы смежности и матрицы инцидентности
\end{remark}

\begin{definition}
	\highlight{$deg (V)$}(степень вершины) --- количество соседей $V$ в графе
\end{definition}

\begin{theorem}[О рукопожатиях]
	В графе $G = (V,E)$ число ребер $ \left| E \right| = \frac{1}{2} \sum_{v \in k} deg(v) $
\end{theorem}
\begin{proof}
	Пусть $x$ --- количество единиц в матрице инцидентности.

	В каждом столбце две единицы. С одной стороны $x = 2S$.

	С другой стороны, в каждой строке ровно $deg(v_{i})$ единиц.

	Т.е. \[
		\sum_{v\in V}^{} deg(V) = X = 2S 	
	\] 
\end{proof}

\begin{definition}
	\highlight{Подграф} $G'= (V'. E')$ графа $G = (V,E)$ --- это граф, у которого $V' \subseteq V$ и $E' \subseteq E$ 
\end{definition}

\begin{definition}
	\highlight{Порожденный подграф $G' = (V', E')$} графа $G = (V, E)$ --- это такой подграф $G$, что вместе
	с $V'$ в $G'$ входят все ребра из $E$, имеющие в качестве концов 2 вершины из $V'$
\end{definition}

\begin{definition}
	\highlight{G' = (V', E')} --- остовный подграф $G = (V,E)$, если для него $V' = V$ 
\end{definition}

\begin{definition}
	\highlight{Маршрут} из $V_1$ в $V_{k}$ в графе $G = (V,E)$ --- это последовательность вида
	\[
	v_1 e_1 v_2 e_2 \ldots e_{k-1} v_{k}
	\] 
\end{definition}

\begin{definition}
	\highlight{Циклический маршрут}  --- маршрут с $v_1 = v_{k}$
\end{definition}

\begin{definition}
	\highlight{Путь} --- маршрут без повторных ребер и $v_1 \neq v_{k}$ 
\end{definition}

\begin{definition}
	\highlight{Простой путь (цепь)} --- путь без повторных вершин
\end{definition}

\begin{definition}
	\highlight{Простой цикл} --- цикл без пересечений по вершинам
\end{definition}

\begin{definition}
	\highlight{Длина пути (цикла)} --- количество ребер в них 
\end{definition}

\begin{remark}
	Длина цикла --- $C_{k}$, длина цепи --- $P_{k}$
\end{remark}

\begin{definition}
	\highlight{Связный граф} --- граф, в котором $ \left(\forall U,V \in V\right) \exists $ маршрут из $U$ в $V$ 
\end{definition}

\begin{definition}
	\highlight{Расстояние между $U$ и $V$} --- длина кратчайшего пути из $U$ в $V$ 
\end{definition}

\begin{definition}
	\highlight{Компонента связности графа $G$} --- максимальный по включению связный подграф графа $G$ 
\end{definition}

\begin{definition}
	\highlight{Дерево} --- связный граф без циклов
\end{definition}

\begin{remark}
	Если исходный граф не содержит циклов и связный, то он --- дерево
\end{remark}

\begin{remark}
	Остовное дерево есть в любом связном графе
\end{remark}

\begin{remark}
	В дереве на $n$ вершин ровно $n-1$ ребро
\end{remark}

\begin{remark}
	В дереве с хотя бы двумя вершинами существует хотя бы $2$ листа
\end{remark}

\begin{definition}
	\highlight{Мост} --- ребро, после удаления которого число компонент связности увеличивается
\end{definition}

\begin{definition}
	\highlight{Точка сочленения} --- вершина, удаление которой приводит к увеличению количества компонент
	связности
\end{definition}

\begin{definition}
	\highlight{Блок (компонента двусвязности)} --- максимальный по включению связный подграф графа без собственных точек сочленения. 
\end{definition}

\begin{remark}
	Изолированная вершина является блоком
\end{remark}

\begin{remark}
	Ребро --- блок $ \iff$ ребро --- мост
\end{remark}

\begin{remark}
	Мост не входит ни в какой цикл
\end{remark}

\begin{remark}
	Ребро-блок --- не является мостом $\implies$ ребро содержится в цикле
\end{remark}

\begin{remark}
	Любые два различных блока либо не пересекаются, либо пересекаются ровно по одной вершине, причём
	эта вершина --- точка сочленения графа
\end{remark}

\begin{proof}
	Пусть блоки пересекаются и есть две общие вершины. Тогда мы получим противоречие с условием максимальности
	блока. Т.е. объединение блоков связное, вершишны связи не являются точками сочленения, т.е. всё это блок.

	Рассмотрим теперь одну вершину пересечения. Пусть это не точка сочленения, т.е. при её удалении из одного
	блока есть ещё путь в другой блок. Рассмотрим теперь два блока и этот путь, получаем связные граф без
	точек сочленения, получили блок и противоречие.
\end{proof}

\begin{remark}
	Если соединить блоки и соответствующие точки сочленения, то получим дерево
\end{remark}

